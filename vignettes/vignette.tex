\documentclass{article}

%\VignetteIndexEntry{Vignette Guidelines}
%\VignetteDepends{}
%\VignetteKeywords{Vignettes}

\usepackage{times}
\usepackage[authoryear,round]{natbib}

\newcommand{\Rfunction}[1]{{\texttt{#1}}}
\newcommand{\Robject}[1]{{\texttt{#1}}}
\newcommand{\Rpackage}[1]{{\textit{#1}}}
\newcommand{\Rclass}[1]{{\textit{#1}}}
\newcommand{\Rmethod}[1]{{\textit{#1}}}
\newcommand{\Rfunarg}[1]{{\textit{#1}}}

\bibliographystyle{plainnat} 

\usepackage{hyperref}
\usepackage{Sweave}
\begin{document}
\Sconcordance{concordance:vignette.tex:vignette.Rnw:%
1 19 1 1 0 217 1}

\title{Writing Vignettes for Bioconductor Packages}
\author{R. Gentleman}
\maketitle{}


This document details some of the important concepts that should be
used and adhered to when writing \textit{vignettes} for Bioconductor
packages. 

A vignette is a navigable document with dynamic content. Sweave
\citep{Leisch2002} provides one system for producing and working with
these documents. A number of tools for dealing with vignettes are
provided by the R package \Rpackage{tools} while others are contained
in the Bioconductor package \Rpackage{DynDoc}. Another source of
information regarding vignettes and their treatment during package
building and checking is provided by the R Extensions manual.

\section*{Some Macros}

In writing \LaTeX\ we recommend using the macros \verb+\texttt{foo}+,
\verb+\textbf{foo}+ and \verb+\textit{foo}+ in preference to the older
style of  \verb+{\em foo}+. An important reason for preferring the
former is that it is easier to do search and replace on them.

It is also recommended that you make use of specific macros for
identifying R functions, objects and macros. The more popular ones are
listed below.
\begin{itemize}
\item \textbf{Robject} \verb+\newcommand{\Robject}[1]{{\texttt{#1}}}+
\item \textbf{Rfunction}
  \verb+\newcommand{\Rfunction}[1]{{\texttt{#1}}}+
\item \textbf{Rpackage} \verb+\newcommand{\Rpackage}[1]{{\textit{#1}}}+
\item \textbf{Rclass} \verb+\newcommand{\Rclass}[1]{{\textit{#1}}}+
\item \textbf{Rmethod} \verb+\newcommand{\Rmethod}[1]{{\textit{#1}}}+
\item \textbf{Rfunarg} \verb+\newcommand{\Rfunarg}[1]{{\textit{#1}}}+
\end{itemize}

Putting these in the header of the \texttt{.Rnw} file makes them
available in \LaTeX.


In addition there are some issues regarding some control statements
for vignettes.  These are typically included as \LaTeX comments at the
very top of the vignette.  For example:

\begin{verbatim}
%\VignetteIndexEntry{Annotation Overview}
%\VignetteDepends{Biobase, genefilter, annotate}
%\VignetteKeywords{Expression Analysis, Annotation}
%\VignettePackage{annotate}
\end{verbatim}

These different identifiers are used as follows:
\begin{itemize}
\item \textbf{VignetteIndexEntry}: will be used as the entry in the
  file \texttt{/inst/doc/00Index.dcf} that is created when the package is
  built. This will then be used by functions such as
  \Rfunction{openVignette} and will be used on menus to identify the
  vignette. So please use short, unique, meaningful names here.
\item \textbf{VignetteDepends}: is a comma separated list of the
  packages that the vignette depends on. At some point tools for
  automatically collecting and installing these packages will be
  developed. 
\item \textbf{VignetteKeywords}: this is a comma separated list of
  keywords and phrases. At some point we hope to provide an ontology
  to select from and some tools for locating vignettes according to
  specified keywords. For now this is mainly informational.
\item \textbf{VignettePackage}: this indicates the name of the package
  that the vignette \textit{belongs} to.
\end{itemize}

In some cases the name of the file (minus the \texttt{.Rnw} suffix) is
used to identify the vignette. To minimize confusion it will be
easiest if the name of the file and the \texttt{VignetteIndexEntry}
are the same.

\section*{Some Words Of Caution}

Please remember that the vignettes get processed when the package is
built. Any functionality that requires interactivity, such as opening
a web page, opening a widget etc. must be placed inside a conditional
statement that is only evaluated when R is interactive.

For example,
\begin{verbatim}
  if(interactive() )
    vExplorer()
\end{verbatim}

So that R only tries to open the \Rfunction{vExplorer} widget when
there is a user to interact with it.

Please also consider whether you want all R commands to be output. In
some settings (particularly developing laboratory materials there are
good arguments for explicitly exhibiting every command but in other
situations some materials should be hidden. For some descriptions it
is more important to explain the important commands and to hide the
details that get the reader to them. Remember, they can always
reconstruct the whole sequence using \Rfunction{vExplorer} and some of
the other tools that have been made available.

Note also that \Rfunction{Sweave} is not very smart; it is basically
just doing text substitution and so it does not know whether your R
code is in a verbatim environment or even located outside of the
\texttt{begin{document}}, \texttt{end{document}} control statements.
    \Rfunction{Sweave} simply searches for the correct syntax (either
    a construct like \verb+<<>>=+ in position one on a line or
    \verb+@+ in position one on a line and interprets it accordingly.

Please use informative labels on the code chunks. This helps others
understand what is going on and the names of the chunks are about all
that the users see when using \Rfunction{vExplorer}. The current
limitations on text output to \Rfunction{vExplorer} also suggest that
you might want to include a few comments in the code chunk to explain,
briefly, what is going on.

\section*{Some Tools}

In \Rpackage{tools} the basic functions for weaving (processing the
document to get dvi or pdf output) and for tangling (extracting the
code chunks to a file) are provided. As part of \Rpackage{DynDoc} we
provide a function called \Rfunction{tangleToR} that extracts the code 
into an R object that can subsequently be used by the reader to
evaluate the different code chunks.

The functionality provided by \Rfunction{vExplorer} will perhaps be
the most use to new users as it allows them to step through the code
chunks sequentially. We will be working on substantial improvements to
this interface for the 1.3 release of Bioconductor.

\section*{Helping Users Find And Use Your Vignette}

Perhaps the hardest task that we face is ensuring that vignettes are
easily available to the users of the system. One way to do that on
Windows (we could do something similar for the Gnome interface), is
to include the following code snippet:
\begin{verbatim}
    if(.Platform$OS.type == "windows" && require(Biobase) && interactive()
        && .Platform$GUI ==  "Rgui"){
        addPDF2Vig("pkgName")
    }
\end{verbatim}

where \texttt{pkgName} is the name of a package. This code checks to
see if it makes sense to try and add a new menu item \texttt{pkgName}
to the menu \texttt{Vignettes} and if so
does just that. For example, \Rfunction{addPDF2Vig("widgetTools")}
will add a new mune item \texttt{widgetTools} to the menu
\texttt{Vignettes}, which has clickable names for all the vignettes
contained by \Rpackage{widgetTools}. Currently, \Rfunction{addPDF2Vig} is
in \Rpackage{Biobase}, meaning that \Rpackage{Biobase} is required for
adding vignettes to a menu. Efforts are being made to remove the dependency on
\Rpackage{Biobase}. 

\section*{Answers To Common Questions And Issues}

The rest of this document will be devoted to answering common
questions and issues that arise when users deal with vignettes.  

\section*{My Vignette Does Not Build Properly}

If a user is having a problem with their vignette not building
properly, there can be several issues involved depending on how they
are building the vignette.  It is important to know the various
mechanisms by which these files are built.  Depending on the
situation, one might be calling \Rfunction{Sweave},
\Rfunction{buildVignettes} or \Rfunction{checkVignettes} (the latter
two call \Rfunction{Sweave}, however).  The \Rfunction{Sweave}
function should report which code chunk caused the error and what that
error was (although it might have been due to a bug in a previous code
chunk).  

If the problem is arising during the operations \Rfunction{R CMD
check} or \Rfunction{R CMD build}, then the functions being used are
\Rfunction{checkVignettes} and \Rfunction{buildVignettes}
respectively.  It is important to note that these are run in a
particular computational environment - simply starting up R and
calling these functions will not necessarily produce the same
results.  It is important to start R with the options
\Rfunarg{--vanilla} and \Rfunarg{--quiet} when loading R to start
either of these functions and \Rfunarg{--no-save} for
\Rfunction{buildVignettes} in order to properly recreate the
environment used.  

In all cases the primary task for the user here is to find out what is
wrong with the actual code and solve that problem.  One method that
can be used is trying the function \Rfunction{Stangle} or
\Rfunction{vExplorer} to go through each code chunk interactively
within R.

\section*{My Vignette Builds Too Slowly}

This can happen sometimes when the code chunks in the document are
themselves very computationally intensive.  Since the code chunks are
run when the vignette gets built, a vignette can end up taking an
extremely long time to generate.  If a user finds themselves in this
situation, there are a few easy steps that can be taken to alleviate
the problem.

One solution to this problem is to split the file up into multiple
vignettes, each describing a smaller piece of the overall puzzle.
This might not be feasible for various reasons - the vignette might
have already been dealing with an atomic topic for instance.  Also, if
the user is creating an R package, putting multiple vignettes in the
package will take just as long as having one very large package when
building the package.

Another potential solution is to try and use precomputed bits of code
wherever possible.  These could be stored as data files in a package
or some other similar location and then loaded instantly instead of
having them be actually computed in the code chunks.  Or, similarly,
the author might be able to construct simpler examples (which might
very well be less useful in a 'real world' sense).

\bibliography{BioC}
\end{document}
